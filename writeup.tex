\documentclass[titlepage]{scrartcl}
\usepackage{graphicx}
\usepackage{float}
\usepackage{amsmath}
\addtokomafont{disposition}{\rmfamily}
%opening
\titlehead{BIOL 224L}
\title{Assignment 5}
\date{\today}
\author{Dean Pearce}

\begin{document}
\maketitle
\section{LR Test: Markov Chain v. Multinomial Model}
\subsection{Code}
\begin{verbatim}
%generate markovian sequence
s2 = markov(x, pi, P, 1000); % markovian
% calculate expected dimer frequency multinomial
expect_mn = transpose(pi) * pi;
% calculate expected dimer frequency markov
expect_markov = transpose(transpose(pi).*P);
% count the number of each dimer in the sequence
s2count_dimer = transpose(cell2mat(struct2cell(dimercount(s2))));
% calculate the likelihood of observing s2 under the null hypothesis of multinomial generation
lambda1 = mnpdf(s2count_dimer, reshape(expect_mn,1,[]));
% and the likelihood under the alternate hypothesis of markovian
lambda2 = mnpdf(s2count_dimer, reshape(expect_markov,1,[]));
% calculate the likelihood ratio test statistic
LR = -2 * log(lambda1 / lambda2);
% test that statistic against a chi-squared distribution with 12 df (weighted binomial = 3 df)
chi2pdf(LR, 9)
\end{verbatim}
\subsection{Results}
Running this with \verb+rng(224)+ results in $p=1.96668919911706\text{e}-07$.
\subsection{Interpretation}
This p-value is quite low ($p^2<<.005$), suggesting that the improvement in performance w.r.t. the ability of the model to mimic \verb+s2+ is statistically likely improbable to have happened by random chance.  However, seeing as how \verb+s2+ was created by the same markov model, differing results would be a surprise.  Utilizing the actual DNA sequence could show different results.
\section{dishonest\_casino}
\subsection{Loaded State Runs}
In \verb|h1| the starting position of the L run is 17, and it continues until 39, giving a length of 22.  In \verb|h2| it begins at 7 and continues until 23, giving a length of 16.  The transition matrix, as predicted by \verb|hmmestimate| is as follows:
\[\begin{pmatrix}
	0.944444444444444 & 0.0555555555555556 \\
	0.0625000000000000 & 0.937500000000000
\end{pmatrix}\]
This does largely explain the observed characteristics of L runs; both F and L states are fairly stable, but with a not insignificant chance of flipping.  It is perfectly reasonable for the results of \verb|h1| and \verb|h2| to arise from it.
\subsection{More Volatile Transition Matrix}
In this scenario, the transition matrix \verb|P| is replaced by \verb|P2|, a more volatile matrix which is as follows:
\[
\begin{pmatrix}
	.60 & .40 \\
	.40 & .60
\end{pmatrix}
\]
This, as one would expect, results in shorter L runs: from 3 to 4, 7 to 9, 11 to 23, 43 to 44.  As the state is more likely to transition from one to another, \verb|hmmviterbi|, as an outsider looking in, is not bound by the "restriction" of \verb|P| to predict a state sequence which is possible within a highly constrained state transition matrix.
\subsection{Aggressive Pseudocounts}
Although altering the pseudocounts does alter the predicted transition matrix and emission probablities, it does so to a very slight degree, one which is not significant enough degree as to affect the decoding of the model and alter the predicted state sequence. Additionally, it is notable that predicting the state sequence completely without pseudocounts also does not significantly alter the predicted transition matrix or emission probablities, nor does it alter the predicted state sequence.  Most likely, this is attributable to the input data being of a large enough size as to allow all possible states, and the behavior under such states, to be exhibited; pseudocounts have the most pronounced effect when they display a possibility which was not present in the sample data.
\end{document}