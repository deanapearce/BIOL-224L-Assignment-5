\documentclass[titlepage]{scrartcl}
\usepackage{graphicx}
\usepackage{float}
\usepackage{amsmath}
\addtokomafont{disposition}{\rmfamily}
%opening
\titlehead{BIOL 224L}
\title{Assignment 5}
\date{\today}
\author{Dean Pearce}

\begin{document}
\maketitle
\section{LR Test: Markov Chain v. Multinomial Model}
\subsection{Code}
\begin{verbatim}
%generate markovian sequence
s2 = markov(x, pi, P, 1000); % markovian
% calculate expected dimer frequency multinomial
expect_mn = transpose(pi) * pi;
% calculate expected dimer frequency markov
expect_markov = transpose(transpose(pi).*P);
% count the number of each dimer in the sequence
s2count_dimer = transpose(cell2mat(struct2cell(dimercount(s2))));
% calculate the likelihood of observing s2 under the null hypothesis of multinomial generation
lambda1 = mnpdf(s2count_dimer, reshape(expect_mn,1,[]));
% and the likelihood under the alternate hypothesis of markovian
lambda2 = mnpdf(s2count_dimer, reshape(expect_markov,1,[]));
% calculate the likelihood ratio test statistic
LR = -2 * log(lambda1 / lambda2);
% test that statistic against a chi-squared distribution with 12 df (weighted binomial = 3 df)
chi2pdf(LR, 9)
\end{verbatim}
\subsection{Results}
Running this with \verb+rng(224)+ results in $p=1.96668919911706\text{e}-07$.
\subsection{Interpretation}
This p-value is quite low ($p^2<<.005$), suggesting that the improvement in performance w.r.t. the ability of the model to mimic \verb+s2+ is statistically likely improbable to have happened by random chance.  However, seeing as how \verb+s2+ was created by the same markov model, differing results would be a surprise.  Utilizing the actual DNA sequence could show different results.
\section{dishonest\_casino}
\subsection{Loaded State Runs}
In \verb|h1| the starting position of the L run is 17, and it continues until 39, giving a length of 22.  In \verb|h2| it begins at 7 and continues until 23, giving a length of 16.  The transition matrix, as predicted by \verb|hmmestimate| is as follows:
\[\begin{array}{cc}
	0.944444444444444 & 0.0555555555555556 \\
	0.0625000000000000 & 0.937500000000000
\end{array}\]
This does largely explain the observed characteristics of L runs; both F and L states are fairly stable, but with a not insignificant chance of flipping.  It is perfectly reasonable for the results of \verb|h1| and \verb|h2| to arise from it.
\subsection{More Volatile Transition Matrix}
In this scenario, the transition matrix \verb|P| is replaced by \verb|P2|, a more volatile matrix which is as follows:
\[
\begin{array}{cc}
	.80 & .20 \\
	.10 & .90
\end{array}
\]
\end{document}